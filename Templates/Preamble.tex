%\documentclass[a4paper, 11pt, dvipdfm]{article}
\usepackage[BoldFont,SlantFont,CJKnumber]{xeCJK}%配合xetex >0.997解决

%%%%%%%%%%%%%日常所用宏包、通通放在一起%%%%%%%%%%%%%%%%%%%%%%%%%%%%
\usepackage[top=2.5cm, bottom=3cm, left=2cm, right=2cm]{geometry}                               
                                     % 控制页边距
\usepackage{enumerate}               % 控制项目列表
\usepackage{multicol}                % 多栏显示

\usepackage[%
    pdfstartview=FitH,%
    CJKbookmarks=true,%
    bookmarks=true,%
    bookmarksnumbered=true,%
    bookmarksopen=true,%
    colorlinks=true,%
    citecolor=blue,%
    linkcolor=blue,%
    anchorcolor=green,%
    urlcolor=blue%
]{hyperref}							 % PDF书签

\usepackage{titlesec}                % 控制标题
\usepackage{titletoc}                % 控制目录
\usepackage{type1cm}                 % 控制字体大小
\usepackage{indentfirst}             % 首行缩进,用\noindent取消某段缩进
%\usepackage{bbding}                  % 一些特殊符号
\usepackage{cite}                    % 支持引用
\usepackage{color,xcolor}            % 支持彩色文本、底色、文本框等
\usepackage{latexsym}                % LaTeX一些特殊符号宏包
\usepackage{amsmath}                 % AMS LaTeX宏包
\usepackage{bm}                      % 数学公式中的黑斜体
\usepackage{relsize}                 % 调整公式字体大小:\mathsmaller, \mathlarger

\usepackage[square]{natbib}
%\makeindex                          % 生成索引

%%%%%%%%%%%%%%%%%%%%%%%%%基本插图方法%%%%%%%%%%%%%%%%%%%%%%%%%%%
\usepackage{graphicx}                % 图形宏包
% \begin{figure}[htbp]               % 控制插图位置
%   \setlength{\abovecaptionskip}{0pt}   
%   \setlength{\belowcaptionskip}{10pt}
                                     % 控制图形和上下文的距离
%   \centering                       % 使图形居中显示
%   \includegraphics[width=0.8\textwidth]{CTeXLive2008.jpg}
                                     % 控制图形显示宽度为0.8\textwidth
%   \caption{CTeXLive2008安装过程} \label{fig:CTeXLive2008}
                                     % 图形题目和交叉引用标签
% \end{figure}
%%%%%%%%%%%%%%%%%%%%%%%%%插图方法结束%%%%%%%%%%%%%%%%%%%%%%%%%%%


%%%%%%%%%%%%%%%%%%%%%%%%%fancyhdr设置页眉页脚%%%%%%%%%%%%%%%%%%%%
\usepackage{fancyhdr}                % 页眉页脚
\pagestyle{fancy}                    % 页眉页脚风格
\setlength{\headheight}{15pt}        % 有时会出现\headheight too small的warning
%\fancyhf{}                          % 清空当前页眉页脚的默认设置
%%%%%%%%%%%%%%%%%%%%%%%%%fancyhdr设置结束%%%%%%%%%%%%%%%%%%%%%%%


%%%%%%%%%%%%%%%%%%%%%%%%%listings宏包粘贴源码%%%%%%%%%%%%%%%%%%%%
\usepackage{listings}                % 方便粘贴源代码,部分代码高亮功能
\lstloadlanguages{}                  % 所要粘贴代码的编程语言

%%%%设置listings宏包的一些全局样式%%%%
%%%%参见http://hi.baidu.com/shawpinlee/blog/item/9ec431cbae28e41cbe09e6e4.html%%%%
\lstset{
breaklines=true,  % 这条命令可以让LaTeX自动将长的代码行换行排版
basicstyle=\normalsize{}\ttfamily{}, % print whole listing small
backgroundcolor=\color{black!10!white},
commentstyle=\color{red!50!green!50!blue!50},
columns=fullflexible,
emph={Huang Deheng, Troy},
emphstyle=\color{blue}\bfseries{},
escapeinside=`',					 % 中文逃逸字符
extendedchars=false,                 % 这一条命令可以解决代码跨页时,章节标题,页眉等汉字不显示的问题
frame=shadowbox,                     % 给代码加框
identifierstyle=\color{black}, % nothing happens
keywordstyle=\color{blue!70},
									 % 关键字高亮
language={[ANSI]C},
linewidth=0.8\textwidth{},
numbers=left,						 % 在左边显示行号
numberstyle=\tiny{},
rulesepcolor=\color{red!20!green!20!blue!20}, 
stringstyle=\ttfamily{}, % typewriter type for strings
showstringspaces=false,
stepnumber=5,
xleftmargin=2em, xrightmargin=2em, aboveskip=1em
}
%%%%%%%%%%%%%%%%%%%%%%%%%listings宏包设置结束%%%%%%%%%%%%%%%%%%%%

\DeclareGraphicsExtensions{.eps,.ps,.eps.gz,.ps.gz,.eps.Z,.EPS,.pdf,.PDF}%声明图片后缀

%字体大小
\newcommand{\chuhao}{\fontsize{42pt}{\baselineskip}\selectfont}
\newcommand{\xiaochuhao}{\fontsize{36pt}{\baselineskip}\selectfont}
\newcommand{\yihao}{\fontsize{28pt}{\baselineskip}\selectfont}
\newcommand{\erhao}{\fontsize{21pt}{\baselineskip}\selectfont}
\newcommand{\xiaoerhao}{\fontsize{18pt}{\baselineskip}\selectfont}
\newcommand{\sanhao}{\fontsize{15.75pt}{\baselineskip}\selectfont}
\newcommand{\sihao}{\fontsize{14pt}{\baselineskip}\selectfont}
\newcommand{\xiaosihao}{\fontsize{12pt}{\baselineskip}\selectfont}
\newcommand{\wuhao}{\fontsize{10.5pt}{\baselineskip}\selectfont}
\newcommand{\xiaowuhao}{\fontsize{9pt}{\baselineskip}\selectfont}
\newcommand{\liuhao}{\fontsize{7.875pt}{\baselineskip}\selectfont}
\newcommand{\qihao}{\fontsize{5.25pt}{\baselineskip}\selectfont}

%%%%定义新字体%%%%
\setCJKfamilyfont{song}{AdobeSongStd-Light}                     
%\setCJKfamilyfont{kai}{Adobe Kaiti Std}
\setCJKfamilyfont{hei}{WenQuanYi Zen Hei}
%\setCJKfamilyfont{fangsong}{Adobe Fangsong Std}
%\setCJKfamilyfont{lisu}{LiSu}
%\setCJKfamilyfont{youyuan}{YouYuan}

\newcommand{\song}{\CJKfamily{song}}                       % 自定义宋体
%\newcommand{\kai}{\CJKfamily{kai}}                         % 自定义楷体
\newcommand{\hei}{\CJKfamily{hei}}                         % 自定义黑体
%\newcommand{\fangsong}{\CJKfamily{fangsong}}               % 自定义仿宋体
%\newcommand{\lisu}{\CJKfamily{lisu}}                       % 自定义隶书
%\newcommand{\youyuan}{\CJKfamily{youyuan}}                 % 自定义幼圆
%%%%%%%%%%%%%%%%%%%%%%%%%xeCJK字体设置结束%%%%%%%%%%%%%%%%%%%%%%

%%%%%%%%%%%%%%%%%%%%%%%%%一些关于中文文档的重定义%%%%%%%%%%%%%%%%%

%%%%数学公式定理的重定义%%%%
\newtheorem{example}{例}                                   % 整体编号
\newtheorem{algorithm}{算法}
\newtheorem{theorem}{定理}[section]                         % 按 section 编号
\newtheorem{definition}{定义}
\newtheorem{axiom}{公理}
\newtheorem{property}{性质}
\newtheorem{proposition}{命题}
\newtheorem{lemma}{引理}
\newtheorem{corollary}{推论}
\newtheorem{remark}{注解}
\newtheorem{condition}{条件}
\newtheorem{conclusion}{结论}
\newtheorem{assumption}{假设}

%%%%章节等名称重定义%%%%
\renewcommand{\contentsname}{目录}     
\renewcommand{\indexname}{索引}
\renewcommand{\listfigurename}{插图目录}
\renewcommand{\listtablename}{表格目录}
\renewcommand{\figurename}{图}
\renewcommand{\tablename}{表}
\renewcommand{\appendixname}{附录}

%%%%设置chapter、section与subsection的格式%%%%
\titleformat{\chapter}{\centering\huge}{第\thechapter{}章}{1em}{\textbf}
\titleformat{\section}{\centering\LARGE}{\thesection}{1em}{\textbf}
\titleformat{\subsection}{\Large}{\thesubsection}{1em}{\textbf}

%%%%%%%%%%%%%%%%%%%%%%%%%中文重定义结束%%%%%%%%%%%%%%%%%%%%

\defaultfontfeatures{Mapping=tex-text}
\setromanfont{Ubuntu}% 设置缺省英文字体
\setsansfont{Monaco}
\setmonofont{FreeMono}
\setCJKmainfont[BoldFont={WenQuanYi Zen Hei}]{WenQuanYi Micro Hei}% 设置缺省中文字体

\makeatletter
\newenvironment{tablehere}
{\def\@captype{table}}
{}

\newenvironment{figurehere}
{\def\@captype{figure}}
{}
\makeatother
